La non-degenerazione è una condizione fondamentale. Infatti se ci troviamo in uno stato degenere dell'hamiltoniana, in genere lo stato è combinazione lineare di autostati della parità: non possiede quindi parità definita. In tal caso la media di una coordinata su quello stato non è necessariamente $0$.
Ad esempio se ci troviamo sull'autostato dell'hamiltoniana:
\begin{equation}
\ket{\psi} = \alpha \ket{\psi_+}+\beta\ket{\psi_-}
\end{equation}
dove $\ket{\psi_\pm}$ sono autostati della parità.
\begin{equation}
\avg{\alpha\psi_++\beta\psi_-}{z} = \alpha^*\beta \melem{\psi_+}{z}{\psi_-}+ \alpha\beta^*\melem{\psi_-}{z}{\psi_+}
\end{equation}
La media non va più a 0. Questo accade in alcune molecole in cui si hanno stati degeneri ma con parità opposta.

\subsubsection{Secondo ordine}
La correzione al secondo ordine si calcola con la formula \vref{eq:Correzione al secondo ordine}.
\begin{equation}\label{eq:pert_2_stark}
E_{100}^{(2)} = e^2\xi^2 \left\{ \sum_{n,l,m} \frac{|\melem{\phi_{nlm}}{z}{\phi_{100}}|^2}{E_1^{(0)}-E_n^{(0)}} + \sum_{k} \frac{|\melem{\phi_k}{z}{\phi_{100}}|^2}{E_1^{(0)}-\hbar^2k^2/2m}\right\}
\end{equation}
Il secondo termire della somma è riferito agli stato non legati dell'atomo dei idrogeno, che vanno inseriti per considerare tutto lo spazio degli stati dell'elettrone. Il problema è difficile da risolvere analiticamente, ma possiamo trovare un limite superiore della perturbazione. Scriviamo più genericamente la formula \ref{eq:pert_2_stark} utilizzando generici autostati di energia $E$, continua o discreta, $\ket{\phi_E}$:
\begin{equation}
E_{100}^{(2)} = e^2\xi^2 \sum_{E \neq E_1^{(0)} } \frac{|\melem{\phi_{100}}{z}{\phi_E}|^2}{E_1^{(0)}-E}
\end{equation}
Sappiamo che $E_1^{(0)}$ è l'energia più bassa perchè è quella del livello fondamentale. 
Quindi 
\begin{equation}
E-E_1^{(0)} \ge E_2^{(0)}- E_1^{(0)} \quad \rightarrow \quad \frac{1}{E-E_1^{(0)}} \le \frac{1}{E_2^{(0)}-E_1^{(0)}}
\end{equation}
Possiamo utilizzare questa disuglianza sulla somma totale:
\begin{equation}
-E_{100}^2 \le e^2\xi^2 \frac{1}{E_2^{(0)}-E_1^{(0)}} \sum_E \melem{\phi_{100}}{z}{\phi_E}\melem{\phi_E}{z}{\phi_{100}}
\end{equation}
Nella somma possiamo anche includere $E=E_1^{(0)}$ poichè il denominatore ora non diverge più e $\avg{\phi_{100}}{z}=0$.
Usiamo ora la relazione di completezza per i vettori $\ket{\phi_E}$. quindi 
\begin{equation}
\sum_E \melem{\phi_{100}}{z}{\phi_E}\melem{\phi_E}{z}{\phi_{100}} = \avg{\phi_{100}}{z^2} 
\end{equation}
Possiamo calcolare facilmente questo valore medio:
\begin{align}
\avg{\phi_{100}}{z^2} &= \avg{\phi_{100}}{x^2} = \avg{\phi_{100}}{y^2} \nonumber \\
\rightarrow\;&= \frac{1}{3}\avg{\phi_{100}}{r^2}
\end{align}